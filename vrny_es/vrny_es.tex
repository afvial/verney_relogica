\documentclass{article}
\usepackage{paracol}

\begin{document}

\begin{paracol}{2}
  
  \subsection*{VI}

  \switchcolumn

  \subsection*{VI}

  \switchcolumn
  
  Peripateticorum caput est Aristoteles Stagirites, Platonis auditor, qui Logicam artem fuse persecutus est ut optime de illa meritus habeatur. Sane reliquis qui post fuere, plurimam lucem adtulit. Conscripsit CXXIII libros de re logica, quorum praecipui desiderantur; XVI tantummodo exstant.

  \switchcolumn

  La cabeza de los peripatéticos es Aristóteles el Estagirita, discípulo de Platón, quien el arte lógico extensamente persiguió\footnote{... expuso}, por lo tanto es correcto que sobre ella\footnote{..., sobre la lógica.} se lo considere merecedor\footnote{... o creador}. Claramente produjo muchísima luz en el resto\footnote{... de autores} que después existieron. Escribió 123 libros sobre la cosa lógica, de los cuales los más especiales\footnote{... o importantes} están extraviados; solamente 16 subsisten.

\end{paracol}

\vspace{0.5cm}

\begin{paracol}{2}

  hola que tal

  \switchcolumn

  hola

\end{paracol}

\begin{paracol}{2}

  \subsection*{VIII}

  \switchcolumn

  \subsection*{VIII}

  \switchcolumn

  Post hos uenit Epicurus Atheniensis, qui Xenocratem et Pamphilum Platonicos et Theophrastum peripateticum et Nausiphanem Pyrrhoni discipulum audiuit, a quo Pyrrhoniorum dogmata accepit. Quidquid uero Academici et stoici, Epicureorum obtrectatores, dixerint de Epicuro, id certum nobis est, ueterum neminem cum ipso esse comparandum; adeo praeclara sunt hominis in artem logicam merita. Quod et Cl. Walchius, qui hanc historiam contexuit, profitetur; immo uero illius Canonica, quam refert Laërtius, quale hominis fuerit ingenium, quam acre iudicium, luculenter demonstrat.

  \switchcolumn

  Luego de esos (filósofos) vino Epicuro de Atenas, quien escuchó a los platónicos Jenócrates y Pánfilo de Samos, al peripatético Teofrasto, y a Nausífanes, discípuo de Pirron, por quien acepto los dogmas.

  \switchcolumn


  Sane Epicurus garrulitatis stoicorum pertaesus, et ineptas cauillationes reiiciens, in quibus ipsi impudenter exsultabant, perinde quasi praeter ceteros et eruditione et ingenii acumine pollerent; paucos canones collegit de perspicuitate sermonis, et recto ordine ratiocinandi, eosque sophistis opposuit; quod sibi persuasit nihil aliud requiri, ut tirones, si illos recte perciperent, stoicorum uafris quaestiunculis occurrerent, eorumque iactantiam labefactarent. Et quamquam nec de recte instituenda ratiocinatione, nec de cauendis fallaciis data opera disputet, quod ipsum animaduertendum est, fatendum est tamen suum illi studium feliciter successisse.

\end{paracol}

\end{document}

